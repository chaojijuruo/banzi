%!TEX program = xelatex
%!TEX encoding = UTF-8 Unicode

\documentclass[12pt, a4paper]{article} % A4纸张,小四号字

%---------- 中文支持配置 ----------
\usepackage{ctex} % 中文核心支持
\usepackage{fontspec} % 字体管理
\setmainfont{Times New Roman} % 英文主字体
\setCJKmainfont{SimSun}[BoldFont=SimHei, ItalicFont=KaiTi] % 中文主字体(宋体/黑体/楷体)

%---------- 页面布局 ----------
\usepackage[top=2.5cm, bottom=2.5cm, left=2.5cm, right=2.5cm]{geometry} % A4边距
\usepackage{setspace} % 行距控制
\renewcommand{\baselinestretch}{1.5} % 1.5倍行距
\setlength{\parindent}{2em} % 首行缩进

%---------- 算法伪代码环境 ----------
\usepackage{amsmath}
\usepackage[linesnumbered, ruled, vlined]{algorithm2e} % 带行号、顶部/底部横线
\SetAlgoCaptionSeparator{:} % 中文化标题分隔符
\SetKwInput{KwInput}{输入} % 设置输入说明
\SetKwInput{KwOutput}{输出} % 设置输出说明
\SetKw{Break}{break} % 重定义关键字
\SetKwIF{If}{ElseIf}{Else}{如果}{则}{否则如果}{否则}{结束} % 中文化条件语句
\SetKwFor{While}{当}{时}{结束} % 中文化循环语句

%---------- 表格支持 ----------
\usepackage{booktabs} % 专业表格工具
\usepackage{multirow} % 多行合并
\usepackage{array} % 列格式控制
\newcolumntype{C}{>{\centering\arraybackslash}p{3cm}} % 自定义居中对齐列类型

%---------- 代码高亮 ----------
\usepackage{listings} % 代码块支持
\usepackage{xcolor} % 颜色定义
\lstset{
  basicstyle=\ttfamily\small,
  keywordstyle=\color{blue},
  commentstyle=\color{green!50!black},
  stringstyle=\color{red},
  breaklines=true, % 自动换行
  frame=single, % 单边框
  numbers=left, % 行号在左侧
  numberstyle=\tiny\color{gray},
  tabsize=4 % 制表符宽度
}

%目录
\usepackage{hyperref}
\usepackage{tocloft} % 在 hyperref 后加载
\setcounter{tocdepth}{3} % 显示section和subsection
%\cftpagenumbersoff{section}    % 取消section页码
%\cftpagenumbersoff{subsection} % 取消subsection页码
%---------- 文档开始 ----------
\begin{document}
%\title{文档标题}
%\author{作者}
%\maketitle
\tableofcontents % 目录位置
\newpage
%---------- 算法示例 ----------
% \section{qwq}
% \subsection{pwp}
% \begin{algorithm}[H]
%   \caption{快速排序算法} 
%   \KwInput{无序数组 $A$, 起始索引 $low$, 结束索引 $high$}
%   \KwOutput{升序排列的数组 $A$}
%   \If{$low < high$}{
%     $pivot \leftarrow \text{Partition}(A, low, high)$ \\
%     \text{QuickSort}(A, low, pivot - 1) \\
%     \text{QuickSort}(A, pivot + 1, high)
%   }
%   \tcp*{划分函数}
%   \SetKwFunction{FPartition}{Partition}
%   \SetKwProg{Fn}{Function}{}{}
%   \Fn{\FPartition{$A$, $low$, $high$}}{
%     $pivot \leftarrow A[high]$ \\
%     $i \leftarrow low - 1$ \\
%     \For{$j \leftarrow low$ \KwTo $high - 1$}{
%       \If{$A[j] < pivot$}{
%         $i \leftarrow i + 1$ \\
%         \text{交换}(A[i], A[j])
%       }
%     }
%     \text{交换}(A[i+1], A[high]) \\
%     \Return $i+1$
%   }
% \end{algorithm}

% %---------- 表格示例 ----------
% \begin{table}[htbp]
%   \centering
%   \caption{神经网络超参数设置}
%   \label{tab:hyperparams}
%   \begin{tabular}{cCc}
%     \toprule
%     \textbf{参数类型} & \textbf{参数值} & \textbf{说明} \\
%     \midrule
%     学习率 & 0.001 & Adam优化器初始学习率 \\
%     批大小 & 64 & 每次迭代样本量 \\
%     \multirow{2}{*}{层数} & \multirow{2}{*}{8} & 卷积层:4 \\
%     & & 全连接层:4 \\
%     \bottomrule
%   \end{tabular}
% \end{table}

% %---------- 代码示例 ----------
% \begin{lstlisting}[language=Python, caption=Python示例代码]
% import numpy as np

% # 计算欧氏距离
% def euclidean_distance(x, y):
%     """计算两个向量的欧氏距离"""
%     return np.sqrt(np.sum((x - y)**2))

% # 主函数
% if __name__ == "__main__":
%     a = np.array([1, 2, 3])
%     b = np.array([4, 5, 6])
%     print(f"距离结果:{euclidean_distance(a, b):.2f}")
% \end{lstlisting}
\section{1}
\end{document}